
\documentclass{beamer}
\usetheme{CambridgeUS}
\title{assignment 8} 
\author{sumeeth kumar\\ai21btech11008}
\date{\today}
\logo{\large \LaTeX{}}


\begin{document}
\begin{frame}
    \titlepage 
\end{frame}
\logo{}
\begin{frame}{Outline}
    \tableofcontents
\end{frame}
\section{question}
\begin{frame}{EX:9.5,question}
Q)The random variables $ a $ and $b$ are independent $N(0;\sigma)$ and p is the probability that the process $x(t)=a-bt $ crosses the t axis in the intervel(0,T)
. show that \pi p=$ arc\tan T$.
\end{frame}
\section{solution}
\begin{frame}{solution:}
   the process crosses the $t$ axis at $x(t)=0$, $a-bt=0$,iff $t=\dfrac{a}{b}$\\
   setting $\sigma_1=\sigma_2=\sigma$ and $r=0$, in theorm we get,\\[8pt]
   \bold THEORM:
   \begin{align}
       F_z(z)=\frac{1}{2}+\frac{1}{\pi}arc\tan\frac{\sigma_2z- r\sigma_1}{\sigma_1\sqrt{1-r^2}}
   \end{align}
   substituting the values we get,\\
   \begin{align}
  &{p[0<t<T]=\frac{1}{2}+\frac{1}{\pi}arc\tan\frac{\sigma (T)}{\sigma}-\left(
  \frac{1}{2}+\frac{1}{\pi}arc\tan0\right)}\\[8pt]
  &{=\frac{1}{2}+\frac{1}{\pi}arc\tan(T)-\left(
  \frac{1}{2}+\frac{1}{\pi}arc\tan0\right)}
  \end{align}
    \end{frame} 
    
    
    \begin{frame}{solution}
            \begin{align}
                &{p=\frac{1}{\pi}arc\tan T}\\[8pt]
                  &{p\pi=arc\tan T}
            \end{align}
             so, hence the  given statement i.e \pi p=$ arc\tan T$ is proved 
        \end{frame}
        
        
\end{document}
